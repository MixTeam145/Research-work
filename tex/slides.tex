\documentclass[ucs, notheorems, handout]{beamer}
\usetheme[numbers,totalnumbers,compress, nologo]{Statmod}
\usefonttheme[onlymath]{serif}
\setbeamertemplate{navigation symbols}{}

\include{letters_series_mathbb.tex}

\mode<handout> {
	\usepackage{pgfpages}
	\setbeameroption{show notes}
	\pgfpagesuselayout{2 on 1}[a4paper, border shrink=5mm]
	\setbeamercolor{note page}{bg=white}
	\setbeamercolor{note title}{bg=gray!10}
	\setbeamercolor{note date}{fg=gray!10}
}

\usepackage[utf8x]{inputenc}
\usepackage[T2A]{fontenc}
\usepackage[russian]{babel}
\usepackage{tikz}
\usepackage{ragged2e}

\title[MC-SSA для многомерных временных рядов]{Метод Монте-Карло SSA для многомерных временных рядов}

\author{Потешкин Егор Павлович, гр.20.Б04-мм}

\institute[Санкт-Петербургский Государственный Университет]{%
	\small
	Санкт-Петербургский государственный университет\\
	Прикладная математика и информатика\\
	Вычислительная стохастика и статистические модели\\
	\vspace{1.25cm}
	Отчет по производственной практике (научно-исследовательская работа) (6 семестр)}

\date[Зачет]{Санкт-Петербург, 2023}

\subject{Talks}	

\begin{document}
\begin{frame}[plain]
	\titlepage
	\note{Научный руководитель  к.ф.-м.н., доцент Голяндина\,Н.\,Э.,\\
		кафедра статистического моделирования}
\end{frame}
\setbeameroption{show notes}
\begin{frame}{Введение}
	
	$\tX^{(d)}=(x_1^{(d)}, x_2^{(d)},\ldots, x_{N_d}^{(d)})$ "--- временные ряды длины $N_d$. $\tX=\{\tX^{(d)}\}_{d=1}^D$ "--- $D$-канальный временной ряд с длинами $N_d$.
	Модель: $\tX=\tS+\tR$, где $\tS$ "--- сигнал, а $\tR$ "--- шум.
	\note{
		Текст про введение
	}\end{frame}
\end{document}}